\begin{frame}{Web :: Dynamic font sizes with CSS}
    % \setbeamercolor{normal text}{fg=black,bg=}
    \small


    \only<1-4>{
        \begin{enumerate}[<+-| alert@+>]
            \item Responsive websites usually require font sizes that change with device (or browser window) width. A font size of 20px that works well on a large monitor is unlikely to be suitable for a smaller tablet or a phone.\parb
            \item With media queries, you can set a different font size for each range of device sizes; this is perfectly adequate in many cases. It does have the disadvantage that when resizing a browser, the user will see the font size jumping from, for example, 18px to 16px to 14px. (But it's usually only designers who spend too much time resizing windows.) \parb
            \item For a more fluid result, viewport units are useful (where {\ttfamily 1vw = 1/100} of the window width). So, for example, if you set your css to \textcolor{black}{\ttfamily font-size:2vw;} and your phone is 400px wide, the font will be 8px. If your window width is 1200px, font size will be 24px.\parb
            \item Using viewport units alone may tend to make the fonts too small on small screens and too large on large screens; one can introduce some fixed sizes as well:\newline \textcolor{\thisColor}{\ttfamily\textbf font-size:calc(9px + 1vw);} sets font size at 13px for phone width of 400px and font size of 21px for window size of 1200px. \pars Still not exactly what you want? Maybe \textcolor{black}{\ttfamily font-size:calc(4px + 1.5vw);} \parb
                  %   \seti
        \end{enumerate}
    }
    \only<5>{
        \begin{enumerate}[<+-| alert@+>]
            % \only<...> seems to break the automated counter...
            % \conti
            \setcounter{enumi}{4}
            \item<5 | alert@5> For more precise control, without trial and error that rapidly becomes frustrating, I turned to this excellent \href{https://css-tricks.com/snippets/css/fluid-typography/}{CSS-tricks} page showing examples such as
                \begin{center}
                    \textcolor{black}{\ttfamily font-size: calc(16px + 6 * ((100vw - 320px) / 680));}
                \end{center}
                What are these magic numbers? At the moment, I can tell that, at a screen width of 320px, the font size is 16px. Font size increases smoothly until, at a screen width of 1000px, the font size of 22px. But I probably won't remember how to figure that out next week! The \href{https://css-tricks.com/snippets/css/fluid-typography/}{CSS-tricks} page doesn't show how these numbers are derived\ldots \pars \ldots but it's just some (relatively) simple high-school math. All you need to recall from high-school is the equation of a line in the form:
                \textcolor{black}{$$\frac{y-y_1}{x-x_1}=\frac{y_2-y_1}{x_2-x_1}$$}
                % \seti
        \end{enumerate}
    }
    \only<6->{
        \begin{textblock*}{1\textwidth}(1.5cm, 1cm)
            \cmini{
                \tikz[scale=0.85]{
                    \coordinate (0) at (0,0);
                    \coordinate (X) at (6,0);
                    \coordinate (Y) at (0,3.5);
                    \coordinate (min) at (1,1.25);
                    \coordinate (max) at (5,2.5);

                    \draw[-to](0)--node[below, pos=0.9, yshift=-0.4cm]{\bf view width}(X);
                    \draw[-to](0)--node[left,pos=0.9, xshift=-0.15cm]{\bf font size}(Y);
                    \node[below left] at (0){0};

                    \gettikzxy{(min)}{\minx}{\miny}
                    \gettikzxy{(max)}{\maxx}{\maxy}
                    \gettikzxy{(X)}{\xx}{\xy}

                    \only<6-8,16,17>{
                        \draw[very thick, \thisColor](0,\miny)--(\minx,\miny)--(max)--(\xx,\maxy);
                    }

                    \only<6-7,16,17>{
                        \node[left] at (0,\miny){10px};
                        \draw[dotted](max)--(0,\maxy)node[left]{20px};
                        \draw[dotted](min)--(\minx,0)node[below]{300px};
                        \draw[dotted](max)--(\maxx,0)node[below]{1200px};
                    }
                    \only<8>{
                        \draw[dotted](min)--(\minx,0)node[below]{$x_1$};
                        \node[yshift=-0.2cm,fill=white, inner sep=0.4mm] at (min) {$(x_1,y_1)$};
                    }
                    \only<8-14>{
                        \node[left] at (0,\miny){$y_1$};
                        \draw[dotted](max)--(0,\maxy)node[left]{$y_2$};
                        \draw[dotted](max)--(\maxx,0)node[below]{$x_2$};
                        \node[yshift=0.2cm,fill=white, inner sep=0.4mm] at (max) {$(x_2,y_2)$};
                    }
                    \only<9-14>{
                        \draw[dotted](\minx,\maxy)--(\minx,0)node[below]{$x_1$};
                        \draw[dotted](\maxx,\miny)--(0,\miny);
                        \node[yshift=-0.2cm,fill=white, inner sep=0.4mm] at (min) {$(x_1,y_1)$};
                        \draw[very thick, \thisColor](min)--(max);
                    }
                    \only<9-14>{
                        \coordinate (P) at ($(min)!0.6!(max)$);
                        \gettikzxy{(P)}{\px}{\py}
                        \draw[thick, red] (P)--(\px,0)node[below, black]{\large $ x$};
                        \draw[thick, red] (P)--(0,\py)node[left, black]{\large $ y$};
                    }
                    \only<15>{
                        \node[yshift=-0.2cm,fill=white, inner sep=0.4mm] at (min) {$(300,10)$};
                        \node[yshift=0.2cm,fill=white, inner sep=0.4mm] at (max) {$(1200,20)$};
                        \draw[very thick, \thisColor](min)--(max);
                    }
                }
            }
        \end{textblock*}
    }
    \begin{textblock*}{1\textwidth}(1cm, 5.25cm)
        \only<6-8>{
            \setcounter{enumi}{5}
            \begin{enumerate}
                \setcounter{enumi}{5}
                \item<6- | alert@6> We can graph our desired font size against view width as shown. In this case: view widths less than 300px have a font size of 10px; font sizes grow uniformly from 10px at 300px view width to a font-size 20px at 1200px window width; and, for window sizes over 1200px, the font size remains 20px. How do we achieve this with CSS?
                \item<7-  | alert@7> The constant font sizes below 300px and above 1200px can be easily handled with the clamp function; we'll come back to that later. What is more interesting is the uniformly increasing font size calculation between view widths of 300px and 1200px.
                \item<8| alert@8> Instead of using the fixed numbers, we'll generalise and use variables so we can easily adjust our formula for different required values.
            \end{enumerate}
        }
        \only<9-12>{
            \begin{enumerate}
                \setcounter{enumi}{8}
                \item<9-|alert@9> For now, just focus on the sloped line as a function from view width $x$ to font size $y$.
                \item<10-|alert@10,11,12> The equation of that line is given by:
                    \textcolor{black}{
                        \begin{align*}
                            \uncover<10->{\frac{y-y_1}{x-x_1} & = \frac{y_2-y_1}{x_2-x_1} \\}
                            \uncover<11->{\Rightarrow y-y_1   & = (x-x_1)\cdot\frac{y_2-y_1}{x_2-x_1}\\}
                            \uncover<12>{\Rightarrow y        & = y_1+(x-x_1)\cdot\frac{y_2-y_1}{x_2-x_1}}
                        \end{align*}
                    }

            \end{enumerate}
        }
        \only<13-15>{
            \begin{enumerate}
                \setcounter{enumi}{12}		% first item will be 6				
                \item<13-|alert@13> We have font size $ = y_1+(x-x_1)\cdot\frac{y_2-y_1}{x_2-x_1}$ where $x_1$, $y_1$, $x_2$ and $y_2$ are numbers chosen for our particular design and $x$ is view width. Of course, CSS does not understand $x$ but view width can be represented by 100vw.
                \item<14-|alert@14> Thus,  \textcolor{black}{\ttfamily font-size:calc($y_1$ + (100vw - $x_1$)*($y_2-y_1$)/($x_2-x_1$));}
                \item<15-|alert@15> From our previous example:
                    \begin{center}
                        \textcolor{black}{\ttfamily font-size:calc(10px + (100vw - 300px)*(20px-10px)/(1200px-300px));}
                    \end{center}
                    or, more concisely:
                    \begin{center}
                        \textcolor{black}{\ttfamily font-size: calc(10px + 10 * (100vw - 300px)/900);}
                    \end{center}

            \end{enumerate}
        }
        \only<16>{
            \begin{enumerate}
                \setcounter{enumi}{15}		% first item will be 9				
                \item<16|alert@16> CSS for the complete range of view widths: \cmini{

                        \textcolor{black}{\ttfamily @media screen and (min-width:1200px) \{ \newline
                            html \{ font-size:20px; \} \newline\} } \pars

                        \textcolor{black}{\ttfamily @media screen and (max-width:1200px) \{ \newline
                            html \{ font-size: calc(10px + (100vw - 300px)/900); \} \newline \} } \pars

                        \textcolor{black}{\ttfamily @media screen and (max-width:300px) \{ \newline
                            html \{ font-size:10px; \} \newline\} }
                    }
                    % \item<17|alert@17>Or, better still, as:
                    % \textcolor{black}{\ttfamily \{ code \} }

            \end{enumerate}
        }
        % \only<17>{
        %     \begin{enumerate}
        %         \setcounter{enumi}{16}
        %         \item<17|alert@17> Or, better still, as: \newline
        %             \textcolor{black}{\ttfamily \newline
        %                 html \{ \newline
        %                 font-size: clamp(10px, calc(10px + (100vw - 300px)/900), 20px);  \newline \} }
        %     \end{enumerate}
        % }       

    \end{textblock*}










\end{frame}